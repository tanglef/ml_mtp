\documentclass[10pt,aspectratio=43]{beamer}

\usepackage{graphicx}
\graphicspath{{prebuilt_images/}}
\usepackage[sectionpages,
			rotationcw % clockwise, default is counterclockwise
			]{../sty/beamerthemeGlobalUniNA}

\usepackage[utf8]{inputenc}
\usepackage[english]{babel}
\usepackage[T1]{fontenc}
\usepackage{csquotes}
\usepackage{amsmath,amsfonts,amsthm,amssymb}

\usepackage[scaled]{beramono}
\usepackage[scale=1.05]{AlegreyaSans}
\usepackage{../sty/shortcuts_fb}

\usetikzlibrary{spy}
\usepackage[round]{natbib}
\bibliographystyle{plainnat}


%%%%%%%%%%%%%%%%%%%%%%%%%%%%%%%%%%%%%%%%%%%%%%%%%%%%%%%%%%%%%%%%%%%%%%%%%%%%%%%
% footnote setting: (1), (2), etc.
\usepackage{fnpct}

% Configure style for custom doubled line
\newcommand*{\doublerule}{\hrule width \hsize height 1pt \kern 0.5mm \hrule width \hsize height 2pt}

% Configure function to fill line with doubled line
\newcommand\doublerulefill{\leavevmode\leaders\vbox{\hrule width .1pt\kern1pt\hrule}\hfill\kern0pt }


\newcommand{\mytheorem}[2]{
\doublerulefill\ \framebox{\textbf{#1}}\ \doublerulefill
\vspace{0.1cm}

#2

\doublerulefill
}

\definecolor{javared}{rgb}{0.6,0,0} % for strings
\definecolor{javagreen}{rgb}{0.25,0.5,0.35} % comments
\definecolor{javapurple}{rgb}{0.5,0,0.35} % keywords
\definecolor{javadocblue}{rgb}{0.25,0.35,0.75} % javadoc
\definecolor{marron}{rgb}{0.64,0.16,0.16}
\definecolor{orange_js}{RGB}{230,159,0}

\newcommand{\mybold}[1]{\textcolor{marron}{\textbf{#1}}}
\newcommand{\cRm}[1]{\textsc{\romannumeral #1}}


%%%%%%%%%%%%%%%%%%%%%%%%%%%%%%%%%%%%%%%%%%%%%%%%%%%%%%%%%%%%%%%%%%%%%%%%%%%%%%%
%%%%%%%%%%%%%%%%%%%%%%%%%%%%%%%%%%%%%%%%%%%%%%%%%%%%%%%%%%%%%%%%%%%%%%%%%%%%%%%
% HEADER
%%%%%%%%%%%%%%%%%%%%%%%%%%%%%%%%%%%%%%%%%%%%%%%%%%%%%%%%%%%%%%%%%%%%%%%%%%%%%%%
%%%%%%%%%%%%%%%%%%%%%%%%%%%%%%%%%%%%%%%%%%%%%%%%%%%%%%%%%%%%%%%%%%%%%%%%%%%%%%%

\title[Short title] %shown at the top of frames
{Long title} %shown in title frame
\subtitle{Based on the article of Trevor Hastie 2020}

\date{04-2021} % explicitly set date instead of \today

\author[]%shown at the top of frames
{%shown in title frame
    {Bascou Florent} \\%
	{Lefort Tanguy}%
}

\institute[
]
{% is placed on the bottom of the title page
    University of Montpellier
}

\titlegraphic{% logos are put at the bottom-right part of the page
    \includegraphics[width=2cm]{Logo.pdf}~ % also support multi-logos
    %\includegraphics[width=2cm]{Logo.pdf}~ % up to 3 (after it gets messy)
    %\includegraphics[width=2cm]{Logo.pdf} % if more, combine them in one image.
}

% You can also move it to where you want.
% This displays 3 logos above the title to the left-center-right
%\titlegraphic{%
%  \begin{picture}(0,0)
%    \put(18,165){\makebox(0,0)[rt]{\includegraphics[width=2cm]{Logo.pdf}
%    \hspace{8em}\includegraphics[width=2cm]{Logo.pdf}
%    \hspace{8em}\includegraphics[width=2cm]{Logo.pdf}}}
%  \end{picture}}

\setbeamercolor{itemize subitem}{fg=red}
\setbeamertemplate{itemize subitem}[triangle]

%%%%%%%%%%%%%%%%%%%%%%%%%%%%%%%%%%%%%%%%%%%%%%%%%%%%%%%%%%%%%%%%%%%%%%%%%%%%%%%
%%%%%%%%%%%%%%%%%%%%%%%%       PLAN      %%%%%%%%%%%%%%%%%%%%%%%%%%%%%%%%%%%%%%
%%%%%%%%%%%%%%%%%%%%%%%%%%%%%%%%%%%%%%%%%%%%%%%%%%%%%%%%%%%%%%%%%%%%%%%%%%%%%%%

\begin{document}
\maketitle


%%%%%%%%%%%%%%%%%%%%%%%%%%%
% Table of contents
%%%%%%%%%%%%%%%%%%%%%%%%%%%
\begin{frame}{Content}{}
    \tableofcontents
\end{frame}

%%%%%%%%%%%%%%%%%%%%%%%%%%%%%%%%%%%%%%%%%%%
%%%%%%%%%%%%%% Introduction %%%%%%%%%%%%%%%
%%%%%%%%%%%%%%%%%%%%%%%%%%%%%%%%%%%%%%%%%%%

\section*{Introduction}
\begin{frame}{Introduction}{Ridge regularization}
truc
\end{frame}


\section{Linear model}
\begin{frame}{Ridge egularization}{Why use it?}
    Ordinary least squares:
    \[\hat\beta = \argmin_{\beta} \|y-X\beta\|_2^2 \Longleftrightarrow \hat \beta = (X^\top X)^{-1}X^\top y\]

    Problems that can happen:
    \begin{itemize}
        \item $X^\top X$ may be ill conditioned ($\kappa = \frac{\text{largest singular value}}{\text{smallest singular value}} \gg $ )
        \begin{onlyenv}<2->
            \begin{itemize}
                \item \color{red}{ Translate spectrum by $\lambda$ using $X^\top X + \lambda \mathrm{Id}.$}
            \end{itemize}
        \end{onlyenv}
        \item $p > n$ leads to infinite number of solutions for OLS.
        \begin{onlyenv}<2->
            \begin{itemize}
                \item \color{red}{ Add penalty to recover unicity.}
            \end{itemize}
        \end{onlyenv}
    \end{itemize}
    \begin{onlyenv}<3->
    \begin{block}{Ridge estimator}
        \[ \hat\beta_{ridge} = \argmin_{\beta} \|y-X\beta\|_2^2 \textcolor{red}{+ \lambda \|\beta\|^2_2} \Longleftrightarrow \hat \beta = (X^\top X \textcolor{red}{+ \lambda \mathrm{Id}})^{-1}X^\top y \]
    \end{block}
    \end{onlyenv}
\end{frame}


%%%%%%%%%%%%%%%%%
% Biblio
%%%%%%%%%%%%%%%%%

\begin{frame}
    \bibliography{../references.bib}

\end{frame}


\end{document}
